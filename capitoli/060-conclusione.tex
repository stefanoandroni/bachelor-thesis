\chapter{Conclusione e sviluppi futuri}
\lhead{Conclusione e sviluppi futuri}

L'obiettivo dello stage può considerarsi raggiunto in quanto il framework realizzato risponde ai requisiti che sono stati individuati all'origine. In particolar modo, il framework ha permesso di sviluppare un tool da linea di comando in grado di eseguire automaticamente test parametrici su applicazioni Android utilizzando in input le tuple offuscate. La potenzialità maggiore può essere individuata però nella sua capacità di verificare se il bug atteso è stato riprodotto e quindi se l'applicativo si è comportato nello stesso modo in cui si sarebbe comportato ricevendo in input i dati originali.
\\\\
Lo strumento potrebbe quindi contribuire al raggiungimento dell'obiettivo finale del progetto in cui lo stage si inserisce, cioè la valutazione dell'efficacia delle tecniche di offuscamento in ambito di riproducibilità di bug, dove con efficacia si intende la capacità delle varie tecniche di generare dati offuscati che riproducano il bug originale, cioè il difetto sollevato dall'inserimento dei dati non offuscati. 
\\\\
Lo strumento è sicuramente ancora lontano dall'essere perfetto e ottimale, inoltre non è stata effettuata una validazione che possa considerarsi esaustiva. A tal fine, come attività futura, sarà necessario testare il framework sviluppato durante lo stage con ulteriori applicazioni e tipoligie di bug.
\\\\
Il lavoro effettuato dai colleghi in precedenza è stato fondamentale alla riuscita dello stage e si spera che anche questa relazione possa aiutare nel proseguimento degli studi.

